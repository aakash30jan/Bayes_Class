\documentclass[11pt]{article}
\usepackage{amsmath,amsfonts,amsthm,amssymb}
\usepackage{times}
\usepackage[pdftex]{graphicx}
\usepackage[pdftex,
        colorlinks=true,
        urlcolor=linkblue,     % \href{...}{...} external (URL)
        citecolor=linkred,     % citation number colors
        linkcolor=linknavy,    % \ref{...} and \pageref{...}
        pdftitle={Flame Spread},
        pdfauthor={Andrew Kurzawski},
        pdfsubject={Flame Spread},
        pdfkeywords={UT},
        pdfproducer={pdflatex},
        pagebackref,
        pdfpagemode=UseNone,
        bookmarksopen=true,
        plainpages=false]{hyperref}
\usepackage{pdfsync}
\usepackage{color}
\usepackage{titling}
\usepackage[nottoc,notlof,notlot]{tocbibind} % Put the bibliography and index in the ToC
\usepackage{listings}

\definecolor{linknavy}{rgb}{0,0,0.50196}
\definecolor{linkred}{rgb}{1,0,0}
\definecolor{linkblue}{rgb}{0,0,1}

\setlength{\droptitle}{-4em}     % Eliminate the default vertical space on the title page.
\addtolength{\droptitle}{5cm}   % Only a guess. Use this for adjustment of the title placement.

\setlength{\textwidth}{6.5in}
\setlength{\textheight}{9.0in}
\setlength{\topmargin}{0.in}
\setlength{\headheight}{0.in}
\setlength{\headsep}{0.in}
\setlength{\parindent}{0.25in}
\setlength{\oddsidemargin}{0.0in}
\setlength{\evensidemargin}{0.0in}

% Python Setup
\lstset{
  language=Python,                % choose the language of the code
  basicstyle=\footnotesize,       % the size of the fonts that are used for the code
  numbers=left,                   % where to put the line-numbers
  numberstyle=\footnotesize,      % the size of the fonts that are used for the line-numbers
  stepnumber=1,                   % the step between two line-numbers. If it is 1 each line will be numbered
  numbersep=5pt,                  % how far the line-numbers are from the code
  backgroundcolor=\color{white},  % choose the background color. You must add \usepackage{color}
  showspaces=false,               % show spaces adding particular underscores
  showstringspaces=false,         % underline spaces within strings
  showtabs=false,                 % show tabs within strings adding particular underscores
  frame=single,           % adds a frame around the code
  tabsize=2,          % sets default tabsize to 2 spaces
  captionpos=b,           % sets the caption-position to bottom
  breaklines=true,        % sets automatic line breaking
  breakatwhitespace=false,    % sets if automatic breaks should only happen at whitespace
  escapeinside={\%*}{*)}          % if you want to add a comment within your code
}

% Uncomment lines and make changes to have a header
\newcommand{\code}[1]{ % change to [2]
  % \hrulefill
  % \subsection*{#1}
  \lstinputlisting{#1} % change to {#2}
  \vspace{2em}
}

\title{Quantifying Flame Spread Uncertainty Using Bayesian Methods}
\author{
        \\
        Andrew Kurzawski \\
        Bayesian Statistical Methods \\ 
        Semester Project \\
}
\date{May 3, 2013}

\begin{document}

\maketitle

\clearpage

\pagestyle{plain}


\section{Introduction and Background}

Modeling physical systems requires that we prescribe values for the model parameters based on previous knowledge, correlations, engineering estimations, etc. If we possess expermental data for the system of interest, we can numerically invert for the unknown model parameters. In higher order systems, the solution space may be complicated, and inversion methods can easily get stuck in local minima and maxima. The result of the inversion process is a set of point estimates for the model input parameters. However, it lacks the uncertainty information necessary for activities such as assessing the model's accuracy or conduction a risk assessment.

Bayesian inference offers a means to fit models of physical systems to observed data and calculate the uncertainty in the model parameters. This method assumes that we do not know the true distribution of the input parameters, but we can use observed data to reconstruct it. Given a scenario or collection of scenarios with observed data, Bayesian methods allow us to sample from the true distribution of the data.

Prediction of fire spread is a major concern to wildland fire fighters who use this information construct tactical plans for supressing and controlling fires. This involves the allocation of resources (personal, command stations, airtankers, etc.) at the appropriate time and location in situations where a poor prediction could cause property damage and put fire figheres and residents lives at risk.

Research efforts in wildland fire modeling have produced models for fire behaviour that range in complexity from algebraic equations to full fluid dynamics and combustion models. This project will focus on the Rothermal flame spread model (ref) that was derived from a combination of fire experiment correlations and physical principles. The result is a set of algebraic equations with eleven input parameters that can be computed quickly. Useing this model with Bayesian methods is equivocal to conducting a Bayesian regression where the regression parameters each have a physical meaning, and can therefore be easily examined for non-physical predictions.

The scenario used for this study involves a large scale (~30 acres) field burn conducted in conjunction with the Texas Forest Service in the Spring of 2011 before Texas saw one of the worst fire seasons on record. We instrumented the field with twenty data loggers to track flame spread and temperature.   

Describe experiment, field size, data loggers, measured quantities.


\section{Flame Spread Model Overview}

Lay out the equations as succinctly as possible. Cite Rothermal. 11 parameters, system of algebriac equations.


\section{Bayesian Model}

Describe data, spread, moisture etc.

Describe unknown parameters. describe priors (uniform for first cut, if time allows consider estimate betas or normals from literature(expert opinions))

Paste in some code maybe. ref appendix


\section{Results and Analysis}

Describe tuning process. Aim for good mixing and low autocorrelation.

Posterior output. Did it fit? Are the values physically significant? What is the uncertainty? What about outliers (check estimated mc data points)? 

What is the physical interpretation of this uncertainty (ie maybe do a simple calculation to see the time frame in which fire could spread over a whole field to give a sense of scale)? Compare posteriors of parameters to the naive sensitivity approach from firetech paper. 

(use this to set up firetech paper) Fire Dynamics Simulator (FDS) is a computiational fluid dynamics code that contains a package for wildland fire spread (ref fds user guide).

Flame spread versus moisture content to show uncertainty and 95 percent probability interval. Explain use for extrapolating to other moisture contents, more data will decrease uncertainty. Show plots from different models. make analogy to huricane forecasting bands with 95 percent confidence interval.

Conclude with possibility of extension to more complicated forecasting models for resource allocation.

\clearpage
\appendix
\section{Appendix: PyMC Model Code}

% \bibliographystyle{unsrt}
% \bibliography{./flameBayesBiblio}

\end{document}
